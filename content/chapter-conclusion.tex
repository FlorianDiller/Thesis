% !TEX root = ../thesis-example.tex
%
\chapter{Final Conclusion}
\label{chap:conclusion}

This thesis introduced a selection of interactive educational approaches along the reality-virtuality continuum.
The presented approaches stretch from rather conventional web applications to novel and innovative augmented reality skill learning systems.
They represent an important addition for the research community regarding educational applications in mixed reality.

\section{Research Questions \label{sec:questions}}
In the following, we want to revisit the research questions we specified in \autoref{chap:concepts}:

\textbf{What impact do experiential education technologies have on students' performance?}\\
In \autoref{chap:ExGoer}, we presented a holistic approach to implementing experiential educational applications into existing schedules.
Not only are the created materials individually adaptable for different learning scenarios and schedules, but they are also published under an open source license.
Finally, the experiential education applications were applied in lectures during a study.
Participants repeating topics with our applications performed better when taking a quiz about the topic.

\textbf{What types of visualizing feedback for motor skill learning exist in \acrlong{mr} and how can they be classified?}\\
We surveyed the existing literature for visual cues regarding feedback for motor skill learning in \autoref{chap:visualCueSurvey}.
Additionally, visual cues for feedback in \acrshort{mr} were identified.
Subsequently, we classified and analyzed the literature regarding visual cues, technology and more.

\textbf{How can superimposed humanlike avatars be registered to facilitate their comparison for motor skill learning?}\\
We introduced a method for avatar registration in \autoref{chap:registration}.
The best results were achieved when registering the avatars vertically considering where they connect to the environment (e.g. feet when standing, hands when doing pull ups).
In addition, the avatars had to be registered in the horizontal plane.
Here, the best results yielded a registration at the spine, closest to the limbs used for the vertical registration.
It facilitates understanding of feedback for motor skill learning, but may find application in other areas like medicine or animation.
However, there were a few scenarios that needed further assessment.

\textbf{How can viewpoint selection take motion feedback into consideration?}\\
Existing research approaches did not take visual feedback into consideration for viewpoint selection.
In \autoref{chap:viewpoint}, we proposed a novel technique for viewpoint selection.
This section is not only real-time suitable and therefore faster than methods in the literature, it also incorporates visual feedback and is favored by the participants of a user study we conducted.

\textbf{How can \acrlong{mr} systems be leveraged to support in-situ feedback for motor skill learning?}\\
In \autoref{chap:omnipresent} we introduced a novel \acrshort{mr} skill learning system.
The system provided a smooth and non irritating positioning on an \acrshort{ar} \acrshort{hmd} and adapted to different szenarios depending on the exercise to ensure an omnipresent feedback.
It did not only provide the feedback in a more comfortable manner to the user, we could also detect no significant disadvantages to conventional \acrshort{rmd}s in a user study.
Furthermore, users were able to identify errors in exercises significantly better.

\section{Future Work}
\label{sec:conclusion:future}

Although this thesis explored many points on the reality-virtuality continuum, there is still much to discover.
The work at hand exemplified the implementation of interactive learning applications across various technologies.
However, it lies beyond the scope of this thesis to cover the entire reality-virtuality continuum.

Consequently, an opportunity to expand the work of this thesis would consist of transferring the approaches to different \acrshort{mr} technologies.
For example, transferring the experiential learning applications presented in \autoref{chap:ExGoer} could yield different results in terms of student performance.
Furthermore, the \acrshort{ar} system of section \autoref{chap:omnipresent} could be converted into an \acrshort{vr} application.
This might require new feedback types and certainly carry new challenges.

Moreover, it is possible to deepen the research concerning motor skill learning.
For example, it is possible to shed light on the feedback for specific body parts.
In particular, the hands come to mind, as a lot of skills are closely linked to hands including professional, artisctic, and musical activities.
Additionally, feedback for facial expressions could be promising to persue.
This could potentially find application in speech therapy, medicine, or theatre.
Focusing specific body parts would certainly mean providing much more detailled feedback.
This in itself could bring a variety of new and interesting challenges to overcome.

An interesting field to expand our discoveries of motor skill learning would be professional skill learning.
Although the literature survey in \autoref{chap:visualCueSurvey} included approaches in this field, there is much to discover.
It could be profitable to see which of the identified visual cues carry over into this use case and if there are new ones to identify.
Furthermore, in this context the use of tools becomes increasingly important.
This in itself is a promising and complex topic as the desired feedback might be highly dependend on the type of tool.

educational MR stuff?

