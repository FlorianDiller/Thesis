% !TEX root = ../thesis-example.tex
%
\chapter{Final Conclusion}
\label{chap:conclusion}

This thesis introduced a selection of interactive educational approaches along the reality-virtuality continuum.
The presented approaches stretch from rather conventional web applications to novel and innovative augmented reality skill learning systems. In addition, the predominant part of the methods can be transferred to various technologies and scenarios. Therefore, they represent an important addition for the research community regarding educational applications in mixed reality. 

\section{Research Questions \label{sec:questions}}
In the following, we want to revisit the research questions we specified in \autoref{chap:concepts}:

\textbf{How can interactive applications beneficially be implemented in teaching?}\\
In \autoref{chap:ExGoer}, we presented a holistic approach to implementing interactive educational applications into existing schedules.
Not only are the created materials individually adaptable for different learning scenarios and schedules, but they are also published under an open source license.

\textbf{What is the state of the art for visualizing feedback for motor skill learning in \acrlong{mr}?}\\
We surveyed the existing literature for visual cues regarding feedback for motor skill learning in \autoref{chap:visualCueSurvey}. In addition, we classified and analyzed the literature regarding visual cues, technology and more.

\textbf{How can superimposed humanlike avatars be registered to facilitate motor skill learning?}\\
We introduced a method for avatar registration in \autoref{chap:registration}. It facilitates understanding of feedback for motor skill learning, but may find application in other areas like medicine or animation. However, there were a few scenarios that needed further assessment.

\textbf{How can viewpoint selection take motion feedback into consideration?}\\
As far as we are concerned, the existing research approaches were not able to take visual feedback into consideration for viewpoint selection. In \autoref{chap:viewpoint}, we proposed an innovate technique for viewpoint selection. This section is not only real-time suitable and therefore faster than methods in the literature, it also incorporates visual feedback and is favored by the participants of a user study we conducted.

\textbf{How can \acrlong{mr} systems be leveraged to support in-situ skill learning?}\\
Lastly, in \autoref{chap:omnipresent} we introduced a novel \acrshort{mr} skill learning system. This system does allow for in-situ visual feedback when learning motor skills. It did not only provide the feedback in a more comfortable manner to the user, we could also detect no significant disadvantages to conventional \acrshort{rmd}s in a user study.


\section{Future Work}
\label{sec:conclusion:future}

Although this thesis explored many points on the reality-virtuality continuum, there is still much to discover. The work at hand exemplified the implementation of interactive learning applications across various technologies. It lies beyond the scope of this thesis to cover the entire reality-virtuality continuum.

Future work could apply the presented techniques to novel scenarios or technologies to exploit their full potential. In addition, this thesis contributes to future research through the insights provided in its chapters, especially the literature survey.
