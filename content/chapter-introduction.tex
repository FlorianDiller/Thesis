% !TEX root = ../thesis-example.tex
%
\externaldocument{content/survey}
\externaldocument{content/registration}
\externaldocument{content/viewpoint}
\externaldocument{content/ExGOER}
\externaldocument{content/omnipresent}
\chapter{Introduction}
\label{sec:intro}

\cleanchapterquote{There is a difference between knowing the path and walking the path.}{Morpheus}{The Matrix}

Since the earliest days of computers they were leveraged as a tool for teaching. The use in universities, which serve as both --- research and education facilities --- seems to have established computers being used in both areas. The earliest teaching system was being developed in 1959 with the Programmed Logic for Automatic Teaching Operations --- short PLATO --- at the University of Illinois \cite{cope2023history}. This early date of education technology highlights the need for engaging learning tools. Even in the earliest iterations of the PLATO system, visual learning aids were applied. Schematics and graphs helped students to approach a subject from different angles. Today, the importance of visualization in education is well established. Computer-based visualizations are said to increase motivation and engagement of learners \cite{vavra2011visualization}. In addition to visualization, interaction plays an important role in education \cite{firat2018towards}. Kolb described interaction in 1984 \cite{kolb:1984:experiential} as a key aspect to creating experiences, which he found critical to establish learning. To leverage both these concepts to the fullest, suitable technologies have to be considered carefully.

The reality-virtuality continuum as defined by Milgram et al. \cite{milgram1994arc} represents a spectrum of environments ranging from real-world to entirely virtual, including various technologies like augmented (AR), virtual (VR), mixed reality (MR) and many more. In recent years, the literature regarding AR and VR technologies in education has increased drastically \cite{alansi2023analyzing}. These technologies become increasingly more accessible and accepted. Yet, development, acquirement and application can still be more extensive than conventional technologies. Therefore, web-based learning applications and dedicated learning programs still have their place based on the use case. This thesis will introduce a selection of interactive learning approaches, which represent different points on the reality-virtuality continuum.

\section{Thesis Structure \label{sec:intro:structure}}
Substantial parts of this thesis describe learning in the context of motor skills in mixed reality. To provide the fundamentals for this topic, \autoref{chap:visualCueSurvey} surveys existing literature for visual cues that are used in mixed reality to facilitate skill learning. From initially 131 works, 39 approaches were analyzed. Not only the visualization methods that helped learning were extracted, technologies, use cases and other details of the publications were revealed and interpreted as well.

Delving deeper into one of the more prevalent visual cues for learning motor skills --- superimposed human skeletal avatars --- \autoref{chap:registration} analyzes the best methods for skeleton registration to facilitate a better training of the motions independently from the technology.

Building upon this, \autoref{chap:viewpoint} explores viewpoint selection methods for superimposed avatars in the literature and introduces a novel viewpoint selection technique. Additionally, our technique is evaluated in a user study against the methods found in the literature.

Furthermore, in \autoref{chap:omnipresent} we present a novel AR motor learning system, taking into account the aspects of the preceding chapters. The adaptive system allows for feedback in training scenarios where it might not be possible to provide feedback in a non-injurious way. 

Lastly, \autoref{chap:ExGoer} introduces an interactive application framework for computer graphics education. The holistic modular approach allows for easy adaptation and creation of new open educational ressources for teaching professionals.

\section{Publications}
ProFIL - Programm zur Förderung des Forschungspersonals,
Infrastruktur und forschendem Lernen of HS Worms

ZIM grant 16KN087122 from the \emph{German Federal Ministry for Economic Affairs and Energy}

This work was conducted as part of the ExGOER project, which was funded by the \emph{OpenEdu-RLP} program of \emph{Virtueller Campus Rheinland-Pfalz}.



This work is based on the following publications:
\begin{itemize}
	\item Visual cue based corrective feedback for motor skill training in mixed reality: A survey \cite{diller2022vcb}
	\item Automatic Viewpoint Selection for Interactive Motor Feedback Using Principle Component Analysis \cite{diller2024automatic}
	\item Towards an Optimal Display of Superimposed Avatars for Motor Feedback \cite{}
	\item SkillAR: omnipresent in-situ feedback for motor skill training using AR \cite{diller2024skillar}
	\item Holistic Approach to Modular Open Educational Resources for Computer Graphics \cite{diller2024holistic}
\end{itemize}

\section{Remarks}

\begin{itemize}
	\item "We"
\end{itemize}