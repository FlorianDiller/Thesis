% !TEX root = ../thesis-example.tex
%
\externaldocument{content/survey}
\externaldocument{content/registration}
\externaldocument{content/viewpoint}
\externaldocument{content/ExGOER}
\externaldocument{content/omnipresent}
\pdfbookmark[0]{Abstract}{Abstract}
%\section*{\textcolor{black}{Interactive Visualization within the Reality-Virtuality Continuum to Support Motor Skill Learning and Education}}
%\vspace*{-5mm}
%{\large \tgherosfont M.Sc. Florian Diller}
\chapter*{\textcolor{black}{Abstract}}
\label{sec:abstract}
\vspace*{-10mm}


The spectrum between real and virtual --- as represented by the reality-virtuality continuum according to Milgram --- offers an endless range of compelling opportunities for user interaction.
The work at hand explores novel visualization methods within the reality-virtuality continuum and how they can be leveraged to teach complex concepts in education and motor skill learning.
Moreover, interaction is of great importance for virtual learning.
Especially, so-called experiential learning approaches can utilize interaction to form unique educational experiences for facilitating learning.
To motivate the topic and its underlying concepts, this thesis starts by exploring the educational background.
The concepts \emph{Experiential Learning}, \emph{Skill Learning} and \emph{Situated Learning} are introduced and linked to establish a connection to the overarching goal of teaching through virtual visual support in Mixed Reality.

Drawing on this foundation, a novel framework for education in an academic setting is presented.
The framework teaches complex theoretical concepts from the field of computer graphics through a holistic framework of interactive web applications, slides and quizzes.
A split test experiment with 19 students was conducted.
Among eight topics, two groups were formed: One conventionally repeating the topic with slides, the other with the interactive web application.
Testing the quiz score before and after repeating, the user group supported by the interactive web application improved by a greater extent.
Additionally, the framework was shown to an UX-expert and teaching personnel, receiving positive responses throughout.
For this approach, the scope is held broad, emphasizing visual interactivity, usability and open access.

Subsequently, the scope is narrowed to provide deeper insight into a topic with complex temporal and spatial concepts to convey: \emph{Motor Skill Learning}.
This is achieved by initially providing a comprehensive literature overview of feedback based on visual cues in the field of motor skill learning in Mixed Reality.
The existing literature was analyzed from 2016 to the present, identifying Mixed Reality methods that provide visual cues for corrective feedback.
This process resulted in 39 relevant papers, providing a diverse range of methods.
Moreover, the selected publications were classified according to their visual cues, Mixed Reality technologies and more.
As a result, it is possible to gain insight into the relationships between different aspects of feedback and the ways in which visual feedback is applied in Mixed Reality within the literature.
Additionally, the literature survey highlighted gaps in the research and the understanding of visual cues for motor skill learning in general.

Addressing these gaps, we analyzed one of the more prevalent visual cues in the surveyed literature for motor skill learning - superimposed human avatars.
In this context, when superimposing two human avatars for the purpose of motor feedback, careful attention must be given to their spatial registration, as it forms the basis for identifying differences in posture or movement.
Therefore, a framework was developed to register superimposed human avatars and several exemplary exercises were provided.
Making sure a connection of the avatars to the environment was established facilitated a better understanding of the target exercise in most cases.

Building upon this foundational knowledge, a method for viewpoint selection was developed.
This method considers feedback in real time by highlighting the visual cues depending on the discrepancy of actual and target movement.
Not only does the method provide a thought-out compromise between a pose-optimized and a feedback-optimized viewpoint, it also ensures a smooth camera movement.
To evaluate the method to approaches from the literature, a user study involving 39 individuals was conducted.
Users could select one of four videos, each optimized by a different method, including the one developed.
As a result, the presented method was most frequently selected as being the most informative.
Additionally, the method was computationally faster than previous viewpoint selection methods for human avatars.
Among the methods tested, it was the only one capable of running in real time.

Lastly, the work at hand presents a novel Augmented Reality feedback system, which leverages an Augemented Reality headset to provide visual motor feedback independent of head position during exercises.
As a result, this allows the user to receive motor feedback in a comfortable manner without risking an incorrect exercise execution.
The spatial positioning of the virtual feedback in Augmented Reality is carefully designed to not irritate the user, cover various use cases, and provide a comfortable experience.
Furthermore, the system was evaluated in an in-subject user study with 32 participants, testing the system against conventional methods found in the literature.
For this purpose, the users solved two tasks: One where motor feedback had to be interpreted, and one where a pose had to be mimicked according to visual feedback.
Subsequently, a few structured questions were asked about the experience.
There was no significant disadvantage detected compared to conventional feedback methods considering identification accuracy as well as identification and execution time.
Additionally, the participants could perform the exercises more comfortably and responded very positively to the feedback system.
Moreover, users identified errors in the interpretation task more frequently.

This thesis presents various novel interactive educational methods within the reality-virtuality continuum.
Delving into different aspects of teaching in academic and athletic environments, the work at hand shows several approaches on how to pursue teaching complex concepts in mixed reality through visualization.
\begin{comment}
\newpage
\section*{\textcolor{black}{Zusammenfassung}}

Das Spektrum zwischen Realität und Virtualität – wie es durch das Reality-Virtuality-Kontinuum nach Milgram dargestellt wird – eröffnet ein nahezu unbegrenztes Potenzial für innovative Formen der Nutzerinteraktion.
Die vorliegende Arbeit untersucht neuartige Visualisierungsmethoden innerhalb dieses Kontinuums und deren Anwendungsmöglichkeiten zur Vermittlung komplexer Inhalte im Bildungsbereich sowie beim Lernen von Bewegungen.
Insbesondere spielt Interaktion eine zentrale Rolle für das virtuelle Lernen.
Erfahrungsbasierte Lernansätze (sog. \emph{Experiential Learning}) können Interaktion gezielt nutzen, um einzigartige Lernerfahrungen zu ermöglichen und dadurch den Lernprozess zu fördern.
Zur Heranführung an das Thema und seine theoretischen Grundlagen beginnt diese Arbeit mit einer Betrachtung der pädagogischen Grundlagen.
Dabei werden die Konzepte \emph{Experiential Learning}, \emph{Skill Learning} und \emph{Situated Learning} eingeführt und miteinander in Beziehung gesetzt, um eine Verbindung zum übergeordneten Ziel – der Wissensvermittlung durch virtuelle visuelle Unterstützung in Mixed Reality – herzustellen.

Aufbauend auf diesem Fundament wird ein neuartiges Framework für die akademische Lehre vorgestellt.
Dieses dient der Vermittlung komplexer theoretischer Inhalte aus dem Bereich der Computergrafik mittels eines ganzheitlichen Ansatzes, bestehend aus interaktiven Webanwendungen, Folien und Quizaufgaben.
Zu Evaluationszwecken wurde ein Split-Test mit 19 Studierenden durchgeführt.
Für acht Themen wurden jeweils zwei Gruppen gebildet: Eine Gruppe wiederholte das Thema auf konventionelle Weise mit Folien, die andere mithilfe der interaktiven Webanwendung.
Die Auswertung der Quiz-Ergebnisse vor und nach der Wiederholung zeigte, dass die mit der interaktiven Anwendung unterstützte Gruppe eine höhere Leistungssteigerung erzielte.
Darüber hinaus wurde das Framework einem UX-Experten und Lehrpersonal vorgestellt, wobei das Feedback durchweg positiv ausfiel.
Im Rahmen dieses Ansatzes wurde bewusst ein breiter Fokus gewählt, mit besonderem Augenmerk auf visuelle Interaktivität, Benutzerfreundlichkeit und offenen Zugang.

Anschließend wird der Rahmen fokussiert, um ein Thema mit besonders komplexen räumlich-zeitlichen Strukturen eingehend zu beleuchten: das Bewegungslernen.
Zunächst erfolgt eine umfassende Literaturübersicht zum Einsatz von visuellem Feedback beim Bewegungslernen in Mixed Reality.
Hierfür wurde einschlägige Literatur aus dem Zeitraum von 2016 bis in die Gegenwart analysiert, mit dem Ziel, Methoden zu identifizieren, die visuelle Hinweise zur Korrektur von Bewegungen einsetzen.
Insgesamt konnten 39 relevante Publikationen ermittelt werden, die eine Vielzahl unterschiedlicher Ansätze abbilden.
Diese Arbeiten wurden hinsichtlich ihrer visuellen Hinweise, verwendeter Mixed-Reality-Technologien und weiterer Merkmale klassifiziert.
Dadurch wurde es möglich, Zusammenhänge zwischen verschiedenen Aspekten von Feedback und deren Anwendung in Mixed Reality systematisch zu erfassen.
Gleichzeitig wurden durch die Literaturrecherche bestehende Forschungslücken identifiziert – sowohl hinsichtlich der Wirkung visueller Hinweise als auch ihres Einsatzes im Kontext des Bewegungslernens.

Um sich mit diesen Lücken zu befassen, wurde eine der häufigsten visuellen Hinweise der analysierten Studien näher untersucht: Überlagerte menschliche Avatare.
Bei der Überlagerung zweier menschlicher Avatare zu Feedbackzwecken ist eine exakte räumliche Registrierung essenziell, da sie die Grundlage zur Identifikation von Unterschieden in Haltung oder Bewegung bildet.
Aus diesem Grund wurde ein Framework zur Registrierung überlagerter Avatare entwickelt und mit exemplarischen Übungen ergänzt.
Durch eine konsequente Einbindung der Avatare in die virtuelle Umgebung konnte überwiegend ein besseres Verständnis der Zielbewegung erzielt werden.

Auf dieser Grundlage wurde eine Methode zur Auswahl optimaler Kamerapositionen entwickelt.
Diese berücksichtigt das Feedback in Echtzeit, indem visuelle Hinweise abhängig von der Abweichung zwischen tatsächlicher und Zielbewegung hervorgehoben werden.
Die Methode stellt nicht nur einen durchdachten Kompromiss zwischen einer haltungs- und feedback-orientierten Perspektive dar, sondern gewährleistet zudem eine flüssige Kamerabewegung.
Zur Evaluation dieser Methode wurde eine Nutzerstudie mit 39 Teilnehmenden durchgeführt.
Den Nutzenden wurden vier Videos präsentiert, die jeweils nach unterschiedlichen Methoden – einschließlich der hier entwickelten – optimiert wurden.
Das entwickelte Verfahren wurde am häufigsten als das informativste gewählt.
Zudem war es im Vergleich zu bestehenden Methoden zur Kameraperspektivenwahl bei menschlichen Avataren recheneffizienter – und somit als einziges Verfahren in der Lage, in Echtzeit berechnet zu werden.

Abschließend stellt die Arbeit ein neuartiges Augmented-Reality-Feedbacksystem vor, das mithilfe eines Augmented Reality-Headsets visuelles motorisches Feedback unabhängig von der Kopfposition während Übungen bereitstellt.
Dies ermöglicht komfortables Feedback, ohne das Risiko einer fehlerhaften Bewegungsausführung.
Die räumliche Positionierung des virtuellen Feedbacks wurde mit dem Ziel gestaltet, den Nutzer nicht zu irritieren, verschiedene Anwendungsfälle abzudecken und ein angenehmes Nutzererlebnis zu gewährleisten.
Das System wurde im Rahmen einer In-Subject-Nutzerstudie mit 32 Teilnehmenden gegen etablierte Verfahren aus der Literatur getestet.
Die Probanden bearbeiteten zwei Aufgaben: In einer sollte motorisches Feedback interpretiert werden, in der anderen eine Körperhaltung basierend auf visuellem Feedback nachgeahmt werden.
Im Anschluss wurden strukturierte Fragen zum Nutzererlebnis gestellt.
Im Vergleich zu herkömmlichen Rückmeldungsverfahren konnten keine signifikanten Nachteile hinsichtlich Genauigkeit sowie Dauer der Identifikation und Ausführung festgestellt werden.
Zudem wurde die Ausführung der Übungen als angenehmer empfunden, und das Feedbacksystem wurde durchweg positiv bewertet.
Darüber hinaus konnten während der Interpretationsaufgabe häufiger Fehler korrekt identifiziert werden.

Diese Arbeit präsentiert eine Vielzahl neuartiger interaktiver Konzepte für Lernen innerhalb des Realität-Virtualität-Kontinuums.
Durch die vertiefte Auseinandersetzung mit verschiedenen Aspekten des Lernens in akademischen und sportlichen Kontexten zeigt die Arbeit auf, wie komplexe Inhalte mittels Visualisierung in Mixed Reality didaktisch aufbereitet und vermittelt werden können.
\end{comment}