% !TEX root = ../thesis-example.tex
%
\externaldocument{content/survey}
\externaldocument{content/registration}
\externaldocument{content/viewpoint}
\externaldocument{content/ExGOER}
\externaldocument{content/omnipresent}
\pdfbookmark[0]{Abstract}{Abstract}
\chapter*{Abstract}
\label{sec:abstract}
\vspace*{-10mm}

The spectrum between real and virtual --- as represented by the reality-virtuality continuum --- offers an endless range of compelling opportunities for user interaction. Therefore, the work at hand explores novel visualization methods within the reality-virtuality continuum and how they can be leveraged for education.
Furthermore, interaction is of great importance for virtual learning. Especially experiential learning approaches can utilize interaction to form unique educational experiences to facilitate learning.
Consequently, this work begins by presenting a modular interactive framework for experiential approaches for learning computer graphics. It can be adapted by learners and teachers to apply to specific learning scenarios. Fundamentally, this concept ventures far into the reality end of the continuum, as it uses traditional questionnaires, scripts, and web applications. However, as the approach represents an extendable framework, it can be the base for approaches everywhere between real and virtual.

Frequently, skill learning can be seen as experiential learning. Therefore, it is an exciting domain for interaction between the real and the virtual. In particular, the support with extended reality seems like the perfect fit, as skill learning frequently involves spatial aspects. Visualizing these spatial aspects represents a particular challenge that numerous approaches seek out to overcome. In \autoref{chap:visualCueSurvey} the existing literature is surveyed for visual cues that are used in mixed reality to facilitate skill learning. Not only the visualization methods that helped learning were extracted, technologies, use cases, and additional details of the publications were revealed and interpreted as well.
Delving deeper into one of the more prevalent visual cues for learning motor skills --- superimposed human skeletal avatars --- \autoref{chap:registration} analyzes the best methods for skeleton registration to facilitate a better training of the motions independently of the technology.
Building upon this, \autoref{chap:viewpoint} explores viewpoint selection methods for superimposed avatars in the literature and introduces a novel viewpoint selection technique. Although it was designed for the use in augmented reality systems, it is generally applicable to a wide range of technologies within the reality virtuality continuum. Additionally, our technique is evaluated in a user study against the methods found in the literature.
Lastly, in \autoref{chap:omnipresent} we present a novel augmented reality motor learning system, taking into account the aspects of the preceding chapters. The adaptive system allows for feedback in training scenarios where it might not be possible to provide feedback in a non-injurious way. A user study was conducted to evaluate the system.


\vspace*{20mm}

{\usekomafont{chapter}Zusammenfassung}\label{sec:abstract-diff} \\

\blindtext
