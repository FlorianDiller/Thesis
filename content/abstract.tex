% !TEX root = ../thesis-example.tex
%
\externaldocument{content/survey}
\externaldocument{content/registration}
\externaldocument{content/viewpoint}
\externaldocument{content/ExGOER}
\externaldocument{content/omnipresent}
\pdfbookmark[0]{Abstract}{Abstract}
\chapter*{Abstract}
\label{sec:abstract}
\vspace*{-10mm}

The spectrum between real and virtual --- as represented by the reality-virtuality continuum --- offers an endless range of compelling opportunities for user interaction. Therefore, the work at hand explores novel visualization methods within the reality-virtuality continuum and how they can be leveraged for education.
Furthermore, interaction is of great importance for virtual learning. Especially experiential learning approaches can utilize interaction to form unique educational experiences to facilitate learning.
Consequently, this work begins by presenting a modular interactive framework for experiential approaches for learning computer graphics. It can be adapted by learners and teachers to apply to specific learning scenarios. Fundamentally, this concept ventures far into the reality end of the continuum, as it uses traditional questionnaires, scripts, and web applications. However, as the approach represents an extendable framework, it can be the base for approaches everywhere between real and virtual.

Frequently, skill learning can be seen as experiential learning. Therefore, it is an exciting domain for interaction between the real and the virtual. In particular, the support with extended reality seems like the perfect fit, as skill learning frequently involves spatial aspects. Visualizing these spatial aspects represents a particular challenge that numerous approaches seek out to overcome. In \autoref{chap:visualCueSurvey} the existing literature is surveyed for visual cues that are used in mixed reality to facilitate skill learning. Not only the visualization methods that helped learning were extracted, technologies, use cases, and additional details of the publications were revealed and interpreted as well.
Delving deeper into one of the more prevalent visual cues for learning motor skills --- superimposed human skeletal avatars --- \autoref{chap:registration} analyzes the best methods for skeleton registration to facilitate a better training of the motions independently of the technology.
Building upon this, \autoref{chap:viewpoint} explores viewpoint selection methods for superimposed avatars in the literature and introduces a novel viewpoint selection technique. Although it was designed for the use in augmented reality systems, it is generally applicable to a wide range of technologies within the reality virtuality continuum. Additionally, our technique is evaluated in a user study against the methods found in the literature.
Lastly, in \autoref{chap:omnipresent} we present a novel augmented reality motor learning system, taking into account the aspects of the preceding chapters. The adaptive system allows for feedback in training scenarios where it might not be possible to provide feedback in a non-injurious way. A user study was conducted to evaluate the system.


\vspace*{20mm}

{\usekomafont{chapter}Zusammenfassung}\label{sec:abstract-diff} \\

Das Spektrum zwischen real und virtuell – wie es durch das Realität-Virtualität-Kontinuum dargestellt wird – bietet eine endlose Vielfalt faszinierender Möglichkeiten für die Benutzerinteraktion. Daher untersucht die vorliegende Arbeit neuartige Visualisierungsmethoden innerhalb dieses Kontinuums und deren Potenzial für den Bildungsbereich. Darüber hinaus spielt Interaktion eine entscheidende Rolle für virtuelles Lernen. Insbesondere erfahrungsbasierte Lernansätze können Interaktion nutzen, um einzigartige Bildungserfahrungen zu schaffen und das Lernen zu erleichtern.

Folglich beginnt diese Arbeit mit der Vorstellung eines modularen interaktiven Frameworks für erfahrungsbasierte Ansätze im Bereich des Lernens von Computergrafik. Dieses kann von Lernenden und Lehrenden an spezifische Lernszenarien angepasst werden. Grundsätzlich bewegt sich dieses Konzept stark im realen Bereich des Kontinuums, da es traditionelle Fragebögen, Skripte und Webanwendungen einsetzt. Da es sich jedoch um ein erweiterbares Framework handelt, kann es als Grundlage für Ansätze entlang des gesamten Realität-Virtualität-Kontinuums dienen.

Häufig kann das Erlernen von Fähigkeiten als erfahrungsbasiertes Lernen betrachtet werden. Daher ist dies ein spannendes Anwendungsgebiet für die Interaktion zwischen Realität und Virtualität. Insbesondere die Unterstützung durch erweiterte Realität (XR) scheint hierfür besonders geeignet zu sein, da das Erlernen von Fähigkeiten oft räumliche Aspekte beinhaltet. Die Visualisierung dieser räumlichen Aspekte stellt eine besondere Herausforderung dar, die zahlreiche Ansätze zu lösen versuchen. In Kapitel 4 wird die vorhandene Literatur zu visuellen Hinweisen in gemischter Realität untersucht, die das Erlernen von Fähigkeiten erleichtern sollen. Dabei wurden nicht nur Visualisierungsmethoden extrahiert, die das Lernen unterstützen, sondern auch Technologien, Anwendungsfälle und weitere Details der Veröffentlichungen analysiert und interpretiert.

Ein besonders verbreiteter visueller Hinweis beim Erlernen motorischer Fähigkeiten ist die Überlagerung menschlicher Skelett-Avatare. Kapitel 5 analysiert die besten Methoden zur Registrierung solcher Skelettmodelle, um ein effektiveres Bewegungstraining unabhängig von der verwendeten Technologie zu ermöglichen. Aufbauend darauf untersucht Kapitel 6 Methoden zur Auswahl von Blickwinkeln für überlagerte Avatare in der Literatur und stellt eine neuartige Technik zur Blickwinkelbestimmung vor. Obwohl diese speziell für den Einsatz in Augmented-Reality-Systemen entwickelt wurde, ist sie allgemein für eine Vielzahl von Technologien innerhalb des Realität-Virtualität-Kontinuums anwendbar. Zudem wird die Technik in einer Nutzerstudie mit den in der Literatur gefundenen Methoden verglichen.

Schließlich wird in Kapitel 7 ein neuartiges Augmented-Reality-System für motorisches Lernen vorgestellt, das die Erkenntnisse der vorhergehenden Kapitel berücksichtigt. Das adaptive System ermöglicht Feedback in Trainingsszenarien, in denen eine gefahrlose Rückmeldung sonst möglicherweise nicht möglich wäre. Eine Nutzerstudie wurde durchgeführt, um das System zu evaluieren.

\textcolor{red}{DURCHLESEN UND ANPASSEN!}