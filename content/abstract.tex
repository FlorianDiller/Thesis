% !TEX root = ../thesis-example.tex
%
\externaldocument{content/survey}
\externaldocument{content/registration}
\externaldocument{content/viewpoint}
\externaldocument{content/ExGOER}
\externaldocument{content/omnipresent}
\pdfbookmark[0]{Abstract}{Abstract}
\chapter*{Abstract}
\label{sec:abstract}
\vspace*{-10mm}

The spectrum between real and virtual --- as represented by the reality-virtuality continuum according to Milgram --- offers an endless range of compelling opportunities for user interaction.
The work at hand explores novel visualization methods within the reality-virtuality continuum and how they can be leveraged to teach complex concepts in education and motor skill learning.
Moreover, interaction is of great importance for virtual learning.
Especially so-called experiential learning approaches can utilize interaction to form unique educational experiences for facilitating learning.
To motivate the topic and its underlying concepts, this thesis starts by exploring the educational background.
The concepts \emph{Experiential Learning}, \emph{Skill Learning} and \emph{Situated Learning} are introduced and linked to establish a connection to the overarching goal of teaching through virtual visual support in Mixed Reality.

Drawing on this foundation, a novel framework for education in an academic setting is presented.
The framework teaches complex theoretical concepts from the field of computer graphics through a holistic framework of interactive web applications, slides and quizzes.
A split test experiment with 19 students was conducted.
Among eight topics, two groups were formed: One conventionally repeating the topic with slides, the other with the interactive web application.
Testing the quiz score before and after repeating, suggested that the user group supported by the interactive web application improved more.
Additionally, the framework was shown to an UX-expert and teaching personnel, receiving positive responses throughout.
In this approach, the scope is held broad, emphasizing visual interactivity, usability and open access.

Subsequently, the scope is narrowed to provide deeper insight into a topic with complex temporal and spatial concepts to convey: \emph{Motor Skill Learning}.
This is achieved by initially providing a comprehensive literature overview of feedback based on visual cues in the field of motor skill learning in Mixed Reality.
The existing literature was analyzed from 2016 to the present, searching for AR and VR methods providing visual cues for corrective feedback in Mixed Reality.
This process resulted in 39 relevant papers, providing a diverse range of methods.
Moreover, the selected publications were classified according to their visual cues, Mixed Reality technologies and more.
As a result, it is possible to gain insight into the relationships between different aspects of feedback and the ways in which visual feedback is applied in Mixed Reality within the literature.
Additionally, the literature survey highlighted gaps in the research and therefore understanding of visual cues for motor skill learning in general.

Addressing these gaps, we identified one of the more prevalent visual cues in the surveyed literature for motor skill learning - superimposed human avatars.
In this context, when superimposing two human avatars for motor feedback, it is crucial how they are registered, as this represents the foundation of the differences in the poses or movements.
Therefore, a framework was developed to register superimposed human avatars and several exemplary exercises were provided.
Making sure a connection to the environment was established facilitated a better understanding of the target exercise in most cases.

Building upon this foundational knowledge, a method for viewpoint selection was developed.
This method considers feedback in real time by highlighting the visual cues depending on the discrepancy of actual and target movement.
Not only, does the method provide a thought-out compromise between a pose-optimized and a feedback-optimized viewpoint, it also ensures a smooth camera movement.
To compare the method to methods from the literature, a user study involving 39 individuals was conducted.
Users could select one of four videos, each optimized by a different method, including the one presented.
As a result, the presented method was most frequently selected as being the most informative.
Additionally, the method was computationally faster than previous viewpoint selection methods for human avatars.
Among the methods tested, it was the only one capable of running in real time.

Lastly, the work at hand presents a novel Augmented Reality feedback system for motor learning, which leverages an AR headset to provide visual motor feedback independent of head position.
As a result, this allows the user to receive motor feedback in a comfortable manner without risking an incorrect exercise execution.
The spatial positioning of the virtual feedback in AR is carefully designed to not irritate the user, cover various use cases, and provide a comfortable experience.
In addition, the system was evaluated in an in-subject user study with 32 participants, testing the system against conventional methods found in the literature.
For this purpose, the users solved two tasked: One where feedback had to be interpreted, and one where a pose had to be mimicked according to visual feedback.
Subsequently, a few structured questions were asked about the experience.
There was no significant disadvantage detected compared to conventional feedback methods considering identification accuracy as well as identification and execution time.
Additionally, the participants could perform the exercises more comfortably and responded very positively to the feedback system.
Furthermore, users identified errors in the interpretation task more frequently.

This thesis presents various novel interactive educational methods within the reality-virtuality continuum.
Delving into different aspects of teaching in academic and athletic environments, the work at hand shows several approaches how to pursue teaching complex concepts in mixed reality through visualization.