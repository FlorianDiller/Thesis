% !TEX root = ../thesis-example.tex
%
\externaldocument{content/survey}
\externaldocument{content/registration}
\externaldocument{content/viewpoint}
\externaldocument{content/ExGOER}
\externaldocument{content/omnipresent}
\pdfbookmark[0]{Abstract}{Abstract}
\chapter*{Abstract}
\label{sec:abstract}
\vspace*{-10mm}

The spectrum between real and virtual --- as represented by the reality-virtuality continuum according to Milgram --- offers an endless range of compelling opportunities for user interaction.
The work at hand explores novel visualization methods within the reality-virtuality continuum and how they can be leveraged to teach complex concepts in education and motor skill learning.
In addition, interaction is of great importance for virtual learning.
Especially so-called experiential learning approaches can utilize interaction to form unique educational experiences for facilitating learning.
To motivate the topic and its underlying concepts, the educational background is examined.
The concepts \emph{Experiential Learning}, \emph{Skill Learning} and \emph{Situated Learning} are explored and linked to establish a connection to the overarching goal of teaching through virtual visual support.

Building upon this foundation, a novel framework for education is presented.
The framework teaches complex theoretical concepts from the field of computer graphics through a holistic framework of interactive web applications, slides and quizzes.
An experiment was conducted, suggesting that the user group supported by the interactive web application scoring higher test scores.
Additionally, the framework was shown to an UX-expert and teaching personnel, receiving throughout positive responses.
Here, the scope is held broad, emphasizing visual interactivity, usability and open access.
Subsequently, the scope is narrowed to provide deeper insight into a topic with complex temporal and spatial concepts to convey: \emph{Motor Skill Learning}.
This is achieved by initially providing a comprehensive literature overview of feedback based on visual cues in the field of motor skill learning in Mixed Reality.
Moreover, the selected publications are classified according to their visual cues, Mixed Reality technologies and more.
As a result, the literature survey highlighted gaps in the research and therefore understanding of visual cues for motor skill learning in general.
Addressing these gaps, we developed a framework for registering superimposed human avatars - a prevalent visual cue for motor skill learning according to the literature overview.
Making sure a connection to the environment was established facilitated a better understanding of the target exercise in most cases.
Building upon this foundational knowledge, a method for viewpoint selection was developed.
This method considers feedback in real time, therefore highlighting the visual cues depending on the discrepancy of actual and target movement.
To evaluate the method against methods from the literature, a user study was conducted.
Users chose the presented method most frequently as being the most informative.
Additionally, the method was computationally faster than previous viewpoint selection methods for human avatars.
Lastly, the work at hand presents a novel Augmented Reality feedback system for motor learning.
An Augmented Reality headset allows for visual motor feedback independent of the head position.
The spatial positioning is optimized to provide feedback in an uninterrupted and comfortable manner.
In addition, the system was evaluated in an in-subject user study.
There was no significant disadvantage detected compared to conventional feedback methods found in literature considering identification accuracy as well as identification and execution time.
Furthermore, users identified errors in exercises shown to them more frequently.

This thesis presents several novel interactive educational methods within the reality-virtuality continuum.
Delving into different aspects of teaching in academic and athletic environments, the work at hand shows several approaches how to pursue teaching complex concepts in mixed reality through visualization.