% !TEX root = ../thesis-example.tex
%
\externaldocument{content/survey}
\externaldocument{content/registration}
\externaldocument{content/viewpoint}
\externaldocument{content/ExGOER}
\externaldocument{content/omnipresent}
\pdfbookmark[0]{Abstract}{Abstract}
\chapter*{Abstract}
\label{sec:abstract}
\vspace*{-10mm}

The spectrum between real and virtual --- as represented by the reality-virtuality continuum --- offers an endless range of compelling opportunities for user interaction. Therefore, the work at hand explores novel visualization methods within the reality-virtuality continuum and how they can be leveraged for education.
Furthermore, interaction is of great importance for virtual learning. Especially experiential learning approaches can utilize interaction to form unique educational experiences to facilitate learning.
Consequently, this work begins by presenting a modular interactive framework for experiential approaches for learning computer graphics. It can be adapted by learners and teachers to apply to specific learning scenarios. Fundamentally, this concept is located far in the reality end of the continuum, as it uses traditional scripts, questionnaires, and web applications. However, as the approach represents an extendable framework, it can be the base for applications everywhere between real and virtual.

Frequently, skill learning can be seen as experiential learning. Therefore, it is an exciting domain for interaction between the real and the virtual. In particular, the support with extended reality seems like the perfect fit, as skill learning frequently involves spatial aspects. Visualizing these spatial aspects represents a particular challenge that numerous approaches seek out to overcome. In \autoref{chap:visualCueSurvey} the existing literature is surveyed for visual cues that are used in mixed reality to facilitate skill learning. Not only the visualization methods that helped learning were extracted, technologies, use cases, and additional details of the publications were revealed and interpreted as well.
Delving deeper into one of the more prevalent visual cues for learning motor skills --- superimposed human skeletal avatars --- \autoref{chap:registration} analyzes the best methods for skeleton registration to facilitate a better training of the motions independently of the technology.
Building upon this, \autoref{chap:viewpoint} explores viewpoint selection methods for superimposed avatars in the literature and introduces a novel viewpoint selection technique. Although it was designed for the use in augmented reality systems, it is generally applicable to a wide range of technologies within the reality virtuality continuum. Additionally, our technique is evaluated in a user study against the methods found in the literature.
Lastly, in \autoref{chap:omnipresent} we present a novel augmented reality motor learning system, taking into account the aspects of the preceding chapters. The adaptive system allows for visual feedback in training scenarios where otherwise it might not be possible to provide feedback in a non-injurious way. A user study was conducted to evaluate the system.


\vspace*{20mm}

{\usekomafont{chapter}Zusammenfassung}\label{sec:abstract-diff} \\

Das Spektrum zwischen real und virtuell --- wie es durch das Realitäts-Virtualitäts-Kontinuum (engl. reality-virtuality continuum) dargestellt wird --- bietet eine endlose Spannbreite an beeindruckenden Möglichkeiten für Nutzerinteraktion. Somit untersucht die vorliegende Arbeit neuartige Visuaisierungsmethoden innerhalb des Realitäts-Virtualitäts-Kontinuums und wie diese wirksam für Bildung eingesetzt können. Infolgedessen beginnt diese Arbeit mit der Vorstellung eines modularen, interaktiven Frameworks für erfahrungsbasiertes Lernen in der Computergrafik. Es kann von Lernenden und Lehrenden an spezifische Lernszenarien angepasst werden. Im Wesentlichen kann dieses Konzept weit in Richtung \emph{Realität} des Kontinuums verortet werden, da es herkömmliche Skripte, Frageböden und Webanwendungen nutzt. Da dieser Ansatz jedoch ein erweiterbares Framework darstellt, kann es als Grundlage für Anwendungen überall zwischen real und virtuell dienen.

Häufig kann das Lernen von Fähigkeiten als erlebnisorientiertes Lernen gesehen werden. Daher repräsentiert es ein spannendes Feld für die Interaktion zwischen real und virtuell. Insbesondere die Unterstützung durch Extended Reality wirkt besonders geeignet, da das Erlernen von Fähigkeiten oft räumliche Aspekte aufweist. Diese Aspekte zu Visualisierungen repräsentiert eine besondere Herausforderung, die zahlreiche Forschungsansätze versuchen zu meistern. In Kapitel 4 wird die vorhande Literatur nach in Mixed Reality genutzten visuellen Hinweisen, die das Erlernen von Fähigkeiten fördern, durchsucht. Es wurden nicht nur die Visualisierungsmethoden, die das Lernen unterstützten, extrahiert, Technologien, Anwendungsfälle und weitere Details der Publikationen wurden ebenfalls enthüllt und interpretiert. Kapitel 5 beschäftigt sich eingehender mit einem der verbreiteteren visuellen Hinweisen für das Erlenen von Fähigkeiten --- übereinanderliegende menschliche, skelettähnliche Avatare. Es werden die besten Methoden für die Registrierung von Skeletten zur Förderung des Trainings unabhängig von Technologien analyisert. Darauf aufbauend untersucht Kapitel 6 Methoden für die Blickwinkelauswahl in der Literatur und führt eine neue Methode zur Auswahl von Blickwinkeln ein. Obwohl diese für die Nutzung in Augmented Reality Systemen entwickelt wurde, ist sie allseitig einsetzbar für eine große Bandbreite an Technologien innerhalb des Realitäts-Virtualitäts-Kontinumms. Zusätzlich wurde unsere Methode in einer Nutzersstudie im Vergleich zu den Methoden der Literatur evaluiert. Schließlich präsentieren wir in Kapitel 7 eine neuartiges Augmented Reality System für motorisches Lernen. Dabei berücksichtigen wir die Aspekte der vorangeschrittenen Kapitel. Das adaptive System erlaubt visuelles Feedback in Trainingsszenarien, wo es sonst nicht möglich sein könnte Feedback auf unschädliche Art bereitzustellen. Eien Nutzerstudie wurde durchgeführt, um das System zu evalieren.
