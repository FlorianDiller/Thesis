% !TEX root = ../thesis-example.tex
%
\chapter{Educational Background}
\label{sec:concepts}

Kurt Lewin --- a pioneer of social psychology --- once said 'Learning is more effective when it is an active rather than a passive process.'. Following that philosophy, Kolb presented in 1984 the experiential learning theory~\cite{kolb:1984:experiential}.
This approach features a learning cycle consisting of four learning stages:
\begin{itemize}
    \item Experiencing (Concrete Experience): The learner experiences something here and now which initiates a learning process.
    \item Reflecting (Reflective Observation): During and after the experience observations are made and data is collected.
    \item Thinking (Abstract Conceptualization): The observation and data of the experience is then analyzed and conclusions are drawn.
    \item Acting (Active Experimentation): The learner's behavior is adapted according to conclusions of the analysis to form new experiences. 
\end{itemize}




\textcolor{red}{how does it relate to MR? Where can it help? }

